\useexternalfigure[engineyard-logo][https://www.engineyard.com/images/company/presskit/EY_Logotype_red.png]

\startmode[presentation]
  \setuppapersize[SW][SW]
  \setuplayout[
    topspace=5pt,
    backspace=20pt,
    width=middle,
    height=middle,
    header=130pt,
    footer=14pt,
    footerdistance=3pt,
    bottomspace=1pt
  ]
  \setupcolors[state=start]
  \setuppagenumber[state=stop]

  \setupbackgrounds[page][background=color, backgroundcolor=white]
  
  \usemodule[simplefonts]
  \setmainfont[Source Sans Pro]

  \define[1]\SlideTitle{
    \midaligned{ \tfb #1 }
    \blank[line]
  }

  % Slide macros
  \def\startSlide{\directsetup{slide:start}}
  \def\stopSlide{\directsetup{slide:stop}}

  \startsetups slide:start
    \start
    \switchtobodyfont[24pt]
    \dontleavehmode
  \stopsetups

  \startsetups slide:stop
    \page
    \stop
  \stopsetups

  \setupfootertexts[][@AllanEspinosa]
\stopmode

\startmode[manuscript]

  \setupframedtexts[location=middle, before={\blank[line]},
                    after={\blank[line]}]
  \let\startSlide\startframedtext
  \let\stopSlide\stopframedtext

  \define[1]\SlideTitle{\midaligned{\bf #1}\blank[small]}
\stopmode

\def\startNote{\startmode[manuscript]}

\starttext

\defineoverlay[titlePage][{\externalfigure[kubecon-template.pdf][page=1,width=\pagewidth]}]
\setupbackgrounds[page][background=titlePage]

\startSlide
\setupfooter[state=stop]
\vfill
\startcolor[white]
\SlideTitle{Autoscaling Containers\ldots with Math}
\midaligned{ @AllanEspinosa }
\stopcolor
\vfill
\stopSlide

\defineoverlay[template][{\externalfigure[kubecon-template.pdf][page=2,width=\pagewidth]}]
\setupbackgrounds[page][background=template]

\startSlide
\midaligned{ \externalfigure[https://images-na.ssl-images-amazon.com/images/I/51xkdqGcykL.jpg][width=0.45\textwidth] }
\stopSlide

\startSlide

\SlideTitle{Your Distributed System}

\tfx
\startMPcode

.PS

arrowht = 0.2
arrowwid = 0.1
linethick = 2
linewid = 1.5

"Goal" at Here rjust
line right ->

ellipse "OPS Person" wid 1.5 ht 0.9

line right -> "Twiddle" above

box "Server Farm" wid 1.5 ht 0.9

line right 2 ->; "CPU Utilization" ljust

line from last line.c down 1.5 then left ->

box "Monitoring" wid 1.5 ht 0.9

line from Here to (1st ellipse.s.x, Here.y) then up to 1st ellipse.s -> "Page" ljust

line from 1st box.n up <-
"Traffic" at Here above


.PE

\stopMPcode

\stopSlide

\startNote
Receive load, 

monitoring system receives performance metrics. Depending on the thresholds you
set in Nagios, you get paged. 

You wake up at 2am. SSH to instances (*gasp!*) or update your Chef environment
attributes. Converge chef. Check if alerts clear.

Then finally go to sleep.

Dynamics where in the state of our system evolves and changes over time due to
several factors.
\stopmode

\startSlide

\SlideTitle{Your Aircon}

\tfx
\startMPcode

.PS

arrowht = 0.2
arrowwid = 0.1
linethick = 2
linewid = 1.5

line right "Set Temperature" above rjust ->

ellipse "Thermostat" wid 1.5 ht 0.9

line right -> "Valve" above

box "Coolant" wid 1.5 ht 0.9

line right 2 "Actual Temperature" above ljust ->

line from last line.c down 1.5 then left ->

box "Sensor" wid 1.5 ht 0.9

line from Here to (1st ellipse.s.x, Here.y) then up to 1st ellipse.s ->

line from 1st box.n up <-
"The Weather" at Here above


.PE

\stopMPcode
\stopSlide

\startNote
Notice how everything look the same. Except for an Ops person being woken up to
adjust the thermostat.

You may have to wake up if you didn't set the thermostat properly. But that's a
different problem for now.

{ \em detect } changes in the environment and { \em respond } to them.
\stopmode

\startSlide

\SlideTitle{Autoscaling}

{\tfx
\startMPcode
.PS

arrowht = 0.2
arrowwid = 0.1
linethick = 2
linewid = 1.0

"Target" at Here rjust
line right 1.0 ->

ellipse "HPA" wid 1.5 ht 0.9

line right 2.5 -> "Number of Pods" above

box "RC" wid 1.5 ht 0.9

line right 2 ->; "Utilization" ljust

line from last line.c down 1.5 then left ->

box "Heapster" wid 1.5 ht 0.9

line from Here to (1st ellipse.s.x, Here.y) then up to 1st ellipse.s -> "Utilization" ljust


line from 1st box.n up <-
"Traffic" at Here above


.PE

\stopMPcode
}

\vfill
{\tfxx
\useURL[kube-docs][http://kubernetes.io/docs/user-guide/horizontal-pod-autoscaling/]
\from[kube-docs]
}

\stopSlide

\startSlide

\vfill
\SlideTitle{Control Theory}

\startitemize
  \item influencing dynamical systems
  \item corrections based on feedback loops
  \item math describes effectiveness
\stopitemize

\vfill

\stopSlide

\startSlide

\midaligned{
\externalfigure[https://upload.wikimedia.org/wikipedia/commons/1/1e/Centrifugal_governor.png][height=0.85\textheight]
}

{\tfxx R. Routledge, {\em Discoveries \& Inventions of the Nineteenth Century}, 13th
edition, 1901. }

\stopSlide

\startSlide

\externalfigure[maxwell.pdf][page=1,height=0.90\textheight]

\stopSlide

\startSlide

\SlideTitle{Control Theory}

\startitemize[columns]
  \item Target Output
  \item Input
  \item Output
  \item Disturbance
\stopitemize

\vfill

\tfx
\startMPcode
.PS
arrowht = 0.2
arrowwid = 0.1
linethick = 2
linewid = 2.0

boxwid = 2
boxht = 1.5


"$r(k)$" at Here rjust
line right -> "$+$" above ljust

circle rad 0.18

line right -> "$e(k)$" above

box "$K(z)$"

line right -> "$u(k)$" above

box "$G(z)$"

line right 3.0 ->

"$y(k)$" at Here ljust

move to last line.c
line down boxht  then left (Here.x - 1st circle.s.x)

line -> to 1st circle.s "$-$" above ljust

line from last box.n up 1.0 <-
"$d(k)$" above


.PE
\stopMPcode
\vfill
\stopSlide

\startNote

Feedback control or control theory changes the approach in that the output is
the observed behavior. In this case, CPU utilization. The actual workload of
traffic affects the utilization but we can't do that much about it. You can't
stop people from buying thing in your shopping side. 

Feedback in a more general term is using what we learned so far and apply it to
the next iteration of our planning. The same goes with the discipline in control
theory.

We have a system where we control the input. Results in certain output. We then 
compare it to our desired output and then adjust the input we give to the
system.

In the example in OPS, we have an objective for our system like let's say
maintain 60\% CPU utilization.  Our system receives traffic and registers a
corresponding status of its current CPU utilization.  When some thresholds are
reached in our alerting system like Nagios or PagerDuty, we wake up at 2am.

We look at what's happening and then decide. Do we continue to sleep or do
something about it?  We can add instances.  Next we observe and wait for a while
before we finally go back to bed... And finally take the morning off at work.

Similar to our Thermostat controller as well.

In the next few slides, I will talk about several concepts in control theory
on what makes a good automation system.  We will look at the basic concept of
autoscaling in Kuberenetes.

Control theory has some math behind it to decide what makes a good feedback
loop.

\stopmode

\startSlide

\externalfigure[control-loop.png][width=\textwidth]

\stopSlide

\startSlide
\externalfigure[demo/proportional-input.png][width=0.85\textwidth]

\externalfigure[demo/proportional-output.png][width=0.85\textwidth]

\stopSlide

\startNote

Kubernetes has a basic facility for autoscaling Replicaton controller based on
CPU utilization.  It basically updates the number of instances of your container
based on the rules you set in the HPA API object.

\stopmode

\startSlide

\vfill
\SlideTitle{Linear-Time Invariant Systems}

$$ a \frac{\partial}{\partial t} y(t) + b y(t) = u(t) $$
$$ y(k + 1) = a y(k) + b u(k) $$
\vfill

\stopSlide



\startSlide

\vfill
\SlideTitle{Desired Properties}
\startframedtext[width=0.5\textwidth,location=middle,frame=off]
\startitemize
  \item Stability
  \item Accuracy
  \item Settling time
  \item Overshoot
\stopitemize
\stopframedtext
\vfill

\stopSlide

\startSlide

\SlideTitle{Stability}

\vfill
\startMPcode

u := 800bp / 25;

pickup pencircle scaled 1bp;

drawarrow (0,-5u) -- (20u, -5u);
drawarrow (0,-5u) -- (0, 5u);

pickup pencircle scaled 2bp;
draw (0, 0u) -- (20u, 0u) dashed dashpattern(on 0.5u off 0.6u);

for k=0 upto 18:
  pickup pencircle scaled 8bp;
  y := (-1.08)**(k);
  draw (k*u, y*u);
endfor

pickup pencircle scaled 3bp;
draw ( (0, u)
  for k=0 upto 18:
    -- (k*u, ((-1.08)**k)*u)
  endfor );

\stopMPcode
\vfill

\stopSlide

\startSlide

\vfill
\startMPcode

u := 800bp / 25;

pickup pencircle scaled 1bp;

drawarrow (0,-u) -- (20u, -u);
drawarrow (0,-u) -- (0, 10u);

pickup pencircle scaled 2bp;
draw (0, 3u) -- (20u, 3u) dashed dashpattern(on 0.5u off 0.6u);

for k=0 upto 18:
  pickup pencircle scaled 8bp;
  y := (1.12)**(k);
  draw (k*u, y*u);
endfor

pickup pencircle scaled 3bp;
draw ( (0, u)
  for k=0 upto 18:
    -- (k*u, ((1.12)**k)*u)
  endfor );

\stopMPcode

\vfill
\stopSlide

\startSlide

\vfill
\startMPcode

u := 800bp / 25;

pickup pencircle scaled 1bp;

drawarrow (0,-5u) -- (20u, -5u);
drawarrow (0,-5u) -- (0, 5u);

pickup pencircle scaled 2bp;
draw (0, 0u) -- (20u, 0u) dashed dashpattern(on 0.5u off 0.6u);

for k=0 upto 19:
  pickup pencircle scaled 8bp;
  y := (-1.0)**(k);
  draw (k*u, 3.5y*u);
endfor

pickup pencircle scaled 3bp;
draw ( (0, 3.5u)
  for k=0 upto 19:
    -- (k*u, 3.5*((-1.0)**k)*u)
  endfor );

\stopMPcode
\vfill
\stopSlide

\startSlide

\vfill
\startMPcode

u := 800bp / 25;

pickup pencircle scaled 1bp;

drawarrow (0,-5u) -- (20u, -5u);
drawarrow (0,-5u) -- (0, 5u);

pickup pencircle scaled 2bp;
draw (0, 0u) -- (20u, 0u) dashed dashpattern(on 0.5u off 0.6u);

for k=0 upto 19:
  pickup pencircle scaled 8bp;
  y := (-0.9)**(k);
  draw (k*u, 3.5y*u);
endfor

pickup pencircle scaled 3bp;
draw ( (0, 3.5u)
  for k=0 upto 19:
    -- (k*u, 3.5*((-0.9)**k)*u)
  endfor );

\stopMPcode
\vfill
\stopSlide

\startSlide
\vfill
\startuseMPgraphic{stable-error}

u := 800bp / 25;

pickup pencircle scaled 1bp;

drawarrow (0,-5u) -- (20u, -5u);
drawarrow (0,-5u) -- (0, 2u);

pickup pencircle scaled 2bp;
draw (0, -0.5u) -- (20u, -0.5u) dashed dashpattern(on 0.5u off 0.6u);

for k=0 upto 19:
  pickup pencircle scaled 8bp;
  y := (0.6)**(k);
  draw (k*u, -4.5y*u);
endfor

pickup pencircle scaled 3bp;
draw ( (0, -4.5u)
  for k=0 upto 19:
    -- (k*u, -4.5*((0.6)**k)*u)
  endfor );

\stopuseMPgraphic

\useMPgraphic{stable-error}
\vfill

\stopSlide

\startSlide

\vfill
\SlideTitle{Accuracy}
\useMPgraphic{stable-error}
\vfill

\stopSlide

\startmode[manuscript]

Accuracy - did we reach our desired CPU utilization? How much are we off?  Since
the workload is not fixed, we need a bit of wiggle room to consider our
autoscaling as good enough. Otherwise we will see oscillations of adding and
subtracting instances all the time.

\stopmode

\startSlide

\SlideTitle{Settling time}
\startMPcode
u := 800bp / 25;
v := 700bp / 25;

pickup pencircle scaled 1bp;

drawarrow (0,-3.3v) -- (20u, -3.3v);
drawarrow (0,-3.3v) -- (0, 2v);

pickup pencircle scaled 2bp;
draw (0, 0.5v) -- (20u, 0.5v) dashed dashpattern(on 0.5u off 0.6u);

for k=0 upto 19:
  pickup pencircle scaled 8bp;
  y := (0.6)**(k);
  draw (k*u, -3.0y*v);
endfor

pickup pencircle scaled 3bp;
draw ( (0, -3.0v)
  for k=0 upto 19:
    -- (k*u, -3.0*((0.6)**k)*v)
  endfor );
\stopMPcode
\vfill

\startMPcode
u := 800bp / 25;
v := 700bp / 25;

pickup pencircle scaled 1bp;

drawarrow (0,-3.3v) -- (20u, -3.3v);
drawarrow (0,-3.3v) -- (0, 2v);

pickup pencircle scaled 2bp;
draw (0, 0.5v) -- (20u, 0.5v) dashed dashpattern(on 0.5u off 0.6u);

for k=0 upto 19:
  pickup pencircle scaled 8bp;
  y := (0.8)**(k);
  draw (k*u, -3.0y*v);
endfor

pickup pencircle scaled 3bp;
draw ( (0, -3.0v)
  for k=0 upto 19:
    -- (k*u, -3.0*((0.80)**k)*v)
  endfor );
\stopMPcode


\stopSlide

\startSlide

\vfill
\SlideTitle{Overshoot}
\startMPcode
u := 800bp / 25;

pickup pencircle scaled 1bp;

drawarrow (0,-5u) -- (20u, -5u);
drawarrow (0,-5u) -- (0, 2u);

pickup pencircle scaled 2bp;
draw (0, -0.5u) -- (20u, -0.5u) dashed dashpattern(on 0.5u off 0.6u);


pair peak;
peak := (5u, 2u);

pickup pencircle scaled 8bp;
draw (0, -4.5u);
draw (u, -3.7u);
draw (2u, -2.5u);
draw (3u, -1.7u);
draw (4u, 0.5u);
draw  peak;
draw (6u, 1u);
draw (7u, 0.3u);
draw (8u, 0.1u);
draw (9u, 0);
draw (10u, 0);
draw (11u, 0);
draw (12u, 0);
draw (13u, 0);
draw (14u, 0);
draw (15u, 0);
draw (16u, 0);
draw (17u, 0);
draw (18u, 0);
draw (19u, 0);

pickup pencircle scaled 3bp;
draw (0, -4.5u)
     ..(u, -3.7u)
     .. (2u, -2.5u)
     .. (3u, -1.7u)
     ..(4u, 0.5u)
     .. peak
     ..(6u, 1u)
     ..(7u, 0.3u)
     ..(8u, 0.1u)
     ..(9u, 0)
     ..(10u, 0)
     ..(11u, 0)
     ..(19u, 0);

pickup pencircle scaled 1bp;
drawdblarrow peak - dir(0)*2u -- (5u, -0.5u) - dir(0)*2u;
draw peak-- peak - dir(0)*2.5u;
draw (5u,-0.5u)-- (5u, -0.5u) - dir(0)*2.5u;
label(btex $y$ etex, 1/2 * (peak - dir(0)*2.5u + (5u, -0.5u) - dir(0)*2.5u) );
\stopMPcode
\vfill
\stopSlide

\startSlide

\SlideTitle{Controllers}

\tfxx
\startMPcode
.PS
arrowht = 0.2
arrowwid = 0.1
linethick = 2
linewid = 2.0

boxwid = 2.4
boxht = 1.5


"$r(k)$" at Here rjust
line right 1.5 -> "$+$" above ljust

circle rad 0.18

line right -> "$e(k)$" above

box "$C(z)$"

line right -> "$u(k)$" above

box "$G(z)$"

line right 3 ->

"$y(k)$" at Here ljust

move to last line.c
line down boxht  then left (Here.x - 1st circle.s.x)

line -> to 1st circle.s "$-$" above ljust

.PE
\stopMPcode

$$ e(k) = r(k) - y(k)$$

$$ u(k) = \sum_{m=-\infty}^\infty c(k-m) e(m) $$

\stopSlide

\startSlide

\SlideTitle{Proportional Control}

$ u_P(k) = K_P e(k) $

%\externalfigure[proportional.pdf][width=0.8\textwidth]
\startMPcode
u := 800bp / 25;
v := 600bp / 15;

pickup pencircle scaled 1bp;

drawarrow (0,-5v) -- (20u, -5v);
drawarrow (0,-5v) -- (0, 2v);

pickup pencircle scaled 2bp;
reference := -0.5v;
draw (0, reference) -- (20u, reference) dashed dashpattern(on 0.5u off 0.6u);


pair peak;
peak := (5u, 2v);

yk := -1.6v;
yk1 := 1.3v;
pickup pencircle scaled 8bp;
draw (0, -4.5v);
draw (u, -3.8v);
draw (2u, -2.8v);
draw (3u, yk);
draw (4u, yk1);

pickup pencircle scaled 3bp;
draw (0, -4.5v)
    .. (u, -3.8v)
    .. (2u, -2.8v)
    .. (3u, yk)
    .. (4u, yk1)
    .. (5u, 0.6v)
    .. (6u, -1.0v)
    .. (7u, -1.2v)
    .. (8u, -1.4v)
    .. (9u, -1.4v)
    .. (10u, -1.2v)
    .. (11u, -1.2v)
    .. (12u, -1.0v)
    .. (13u, -1.0v)
    .. (14u, -1.0v)
    .. (15u, -1.0v)
    .. (16u, -1.0v)
    .. (17u, -1v)
    .. (18u, -1v)
    .. (19u, -1v);

pickup pencircle scaled 3bp;

pickup pencircle scaled 1bp;
drawdblarrow (3u, reference) - dir(0)*u -- (3u, yk) - dir(0)*u;
draw (3u, reference)-- (3u, reference) - dir(0)*1.5u;
draw (3u, yk)-- (3u, yk) - dir(0)*1.5u;
label.lft(btex {\smaller $e(k)$} etex, 1/2 * ((3u, reference) + (3u, yk) ) - dir(0)*1.2u );

drawdblarrow (4u, yk1) + dir(0)*u -- (4u, yk) + dir(0)*u;
draw (4u, yk1)-- (4u, yk1) + dir(0)*1.5u;
draw (3u, yk)-- (4u, yk) + dir(0)*1.5u;
label.rt(btex {\smaller $K_pe(k)$} etex, 1/2 * ((4u, yk1) + (4u, yk) ) + dir(0)*2.0u + dir(90)*u);


\stopMPcode

\stopSlide

\startNote

Assuming $y(k + 1) = y(k) + u(k)$.

\stopmode

\startSlide

\SlideTitle{Proportional Control}
\startitemize
  \item inherently inaccurate
  \item $K_P$ increases overshoot and settling time
\stopitemize

\stopSlide

\startSlide

\SlideTitle{Integral Control}
$$ u_I(k) = u_I(k-1) + K_I e(k) $$

\startitemize
  \item reduce steady-state error
  \item increase settling times
\stopitemize

\stopSlide

\startSlide

\SlideTitle{Derivative Control}

$$ u_D(k) = K_D \[ e(k) - e(k-1) \] $$

\startitemize
  \item decrease settling times
  \item sensitive to noise
\stopitemize

\stopSlide

\startSlide

\SlideTitle{Non-linearity}
\externalfigure[nonlinear.pdf][width=0.90\textwidth]

\stopSlide

\startSlide

\SlideTitle{Summary}

\startitemize
  \item iterate on feedback
  \item effectiveness of feedback
  \item linear models go a long way
  \item re-evaluate your models!
\stopitemize

\stopSlide

\startNote
effectiveness - stability, steady state error, settling time, overshoots
\stopmode

\startSlide

\dontleavehmode
\externalfigure[https://images-na.ssl-images-amazon.com/images/I/41iAmXZ8zbL.jpg][height=0.7\textheight]
\externalfigure[https://images-na.ssl-images-amazon.com/images/I/41Lg4IZz+nL.jpg][height=0.7\textheight]

{\tfx
  P. Janert, {\em Feedback Control for Computer Systems}, Sebastapol, CA:
O'Reilly Media, 2014.

  J. Hellerstein, et. al., {\em Feedback Control of Computing Systems},
Hoboken, NJ: John Wiley \& Sons, 2004.
}

\stopSlide

\startSlide

\vfill
\SlideTitle{Thank You!}

\midaligned{@AllanEspinosa}

\vfill
{\tfx
\useURL[lab-notes][https://github.com/aespinosa/control-theory]
\from[lab-notes]
}
\stopSlide

\stoptext
